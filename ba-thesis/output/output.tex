\newcommand{\tones}[1]{
    {\fontspec{Noto Sans CJK JP}#1}
    }


% Options for packages loaded elsewhere



%
\documentclass[stu,a4paper,11pt,floatsintext]{apa7}

% deps for huxtable
\usepackage{array}
\usepackage{graphicx}
\usepackage{siunitx}
\usepackage{colortbl}
\usepackage{multirow}
\usepackage{hhline}
\usepackage{calc}
\usepackage{tabularx}
\usepackage{threeparttable}
\usepackage{wrapfig}
\usepackage{adjustbox}
\usepackage{hyperref}

\usepackage{tabulary}
\usepackage{booktabs}
\usepackage{csquotes}
\usepackage{hyperref}
\usepackage{xcolor}
\usepackage{sectsty}
\usepackage{tikz}
\usepackage{csquotes}
\usepackage{lineno}
\usepackage[section]{placeins}

\captionsetup[figure]{skip=5pt, font=small, labelsep=period}
\linenumbers
% Polar Night
\definecolor{NordDarkBlack}{HTML}{2E3440}     % nord0
\definecolor{NordBlack}{HTML}{3B4252}         % nord1
\definecolor{NordMediumBlack}{HTML}{434C5e}   % nord2
\definecolor{NordBrightBlack}{HTML}{4C566A}   % nord3
% Snow Storm
\definecolor{NordWhite}{HTML}{D8DEE9}         % nord4
\definecolor{NordBrighterWhite}{HTML}{E5E9F0}         % nord5
\definecolor{NordBrightestWhite}{HTML}{ECEFF4}   % nord6
% Frost
\definecolor{NordCyan}{HTML}{8FBCBB}          % nord7
\definecolor{NordBrightCyan}{HTML}{88C0D0}    % nord8
\definecolor{NordBlue}{HTML}{81A1C1}          % nord9
\definecolor{NordBrightBlue}{HTML}{5E81AC}    % nord10
% Aurora
\definecolor{NordRed}{HTML}{BF616A}           % nord11
\definecolor{NordOrange}{HTML}{D08770}        % nord12
\definecolor{NordYellow}{HTML}{EBCB8B}        % nord13
\definecolor{NordGreen}{HTML}{A3BE8C}         % nord14
\definecolor{NordMagenta}{HTML}{B48EAD}       % nord15


\LetLtxMacro\latexincludegraphics\includegraphics
\renewcommand{\includegraphics}[1]{
	\latexincludegraphics[width = \textwidth]{#1}
}

\sectionfont{\color{NordBlack}}  % sets colour of sections
\subsectionfont{\color{NordBlack}}  % sets colour of sections

\usepackage[american]{babel}
\usepackage{fontspec}


%\setmainfont[
%    BoldFont={Meta Pro Medium}
%]{Meta Pro}


%\setmainfont[
%    BoldFont={Gulliver Bold}
%]{Gulliver}




\newenvironment{cslreferences}%
  {}%
  {\par}

\usepackage{lmodern}
\usepackage{amssymb,amsmath}
\usepackage{ifxetex,ifluatex}
\ifnum 0\ifxetex 1\fi\ifluatex 1\fi=0 % if pdftex
  \usepackage[T1]{fontenc}
  \usepackage[utf8]{inputenc}
  \usepackage{textcomp} % provide euro and other symbols
\else % if luatex or xetex
  \usepackage{unicode-math}
  \defaultfontfeatures{Scale=MatchLowercase}
  \defaultfontfeatures[\rmfamily]{Ligatures=TeX,Scale=1}
\fi
% Use upquote if available, for straight quotes in verbatim environments
\IfFileExists{upquote.sty}{\usepackage{upquote}}{}
\IfFileExists{microtype.sty}{% use microtype if available
  \usepackage[]{microtype}
  \UseMicrotypeSet[protrusion]{basicmath} % disable protrusion for tt fonts
}{}
\makeatletter
\@ifundefined{KOMAClassName}{% if non-KOMA class
  \IfFileExists{parskip.sty}{%
    \usepackage{parskip}
  }{% else
    \setlength{\parindent}{0pt}
    \setlength{\parskip}{6pt plus 2pt minus 1pt}}
}{% if KOMA class
  \KOMAoptions{parskip=half}}
\makeatother
\usepackage{xcolor}
\IfFileExists{xurl.sty}{\usepackage{xurl}}{} % add URL line breaks if available
\IfFileExists{bookmark.sty}{\usepackage{bookmark}}{\usepackage{hyperref}}
\hypersetup{
  hidelinks,
  pdfcreator={LaTeX via pandoc}}
\urlstyle{same} % disable monospaced font for URLs

\usepackage{graphicx}
\makeatletter
\def\maxwidth{\ifdim\Gin@nat@width>\linewidth\linewidth\else\Gin@nat@width\fi}
\def\maxheight{\ifdim\Gin@nat@height>\textheight\textheight\else\Gin@nat@height\fi}
\makeatother
% Scale images if necessary, so that they will not overflow the page
% margins by default, and it is still possible to overwrite the defaults
% using explicit options in \includegraphics[width, height, ...]{}
%\setkeys{Gin}{width=\maxwidth,height=\maxheight,keepaspectratio}
% Set default figure placement to htbp
\makeatletter
\def\fps@figure{htbp}
\makeatother
\setlength{\emergencystretch}{3em} % prevent overfull lines
\providecommand{\tightlist}{%
  \setlength{\itemsep}{0pt}\setlength{\parskip}{0pt}}
\setcounter{secnumdepth}{-\maxdimen} % remove section numbering
\ifluatex
  \usepackage{selnolig}  % disable illegal ligatures
\fi


\title{}
\author{}
\date{}




\begin{document}

\maketitle


\renewcommand*\contentsname{}

\setcounter{tocdepth}{3}
\tableofcontents

Statistical analyses provided support for these findings. For the 100 ms
stimulation rate, the three-way ANOVA yielded a significant three-way
interaction effect (\emph{condition} x \emph{stimulus type} x
\emph{electrode locations}; \(F(1,19) = 7.53\), \(p = 0.0130\)) but
failed to reveal main effects for factors \emph{stimulus type}
(\(F(1,19) = 1.05\), \(p = 0.3180\)), \emph{condition}
(\(F(1,19) = 0.83\), \(p = 0.3730\)), and \emph{electrode locations}
(\(F(1,19) = 0.04\), \(p = 0.8520\)). In contrast, for tones presented
at a SOA of 150 ms only the two-way interaction term \emph{stimulus
type} x \emph{electrode locations} had a significant effect
(\(F(1,22) = 20.76\), \(p = 0.0002\)). Mean amplitudes in the MMN
latency window however did not differ for factors \emph{stimulus type}
(\(F(1,22) = 0.32\), \(p = 0.5790\)), \emph{electrode locations} ().

\input{tables/anova_02_full.tex}

Two-way ANOVAs (\emph{condition} x \emph{stimulus type}) were carried
out seperatly for pooled fronto-central and mostoid electrode locations.
For 100 ms tone presenation rate, the \emph{condition} x \emph{stimulus
type} interaction only revealed a significant effect for the
fronto-central electrode cluster (\(F(1,19) = 16.75\), \(p = 0.0006\))
but not for pooled mastoid sites (\(F(1,19) = 2.37\), \(p = 0.1410\))
indicating that the three-way interaction effect condition x stimulus
type x electrode is indeed driven by the amplitude differnces in te
fronto-central electrode locations . Contrary to this, for the 150 ms
presentation rate, main effects for \emph{stimulus type} were
significant for both fronto-central and mastoid sites, suggesting that
there was both a MMN at fronto-central locations as well as a
polarity-reversal at the mastoid electrodes.


  \providecommand{\huxb}[2]{\arrayrulecolor[RGB]{#1}\global\arrayrulewidth=#2pt}
  \providecommand{\huxvb}[2]{\color[RGB]{#1}\vrule width #2pt}
  \providecommand{\huxtpad}[1]{\rule{0pt}{#1}}
  \providecommand{\huxbpad}[1]{\rule[-#1]{0pt}{#1}}

\begin{table}[ht]
\begin{centerbox}
\begin{threeparttable}
\captionsetup{justification=centering,singlelinecheck=off}
\caption{Results of the 3-way ANOVA (condition x stimulus x electrode) for repeated measures conducted on the mean ERP-amplitudes (time window 111 - 161 ms) at electrode Fz (upper section). The significant interaction between the three factors included was further analyzed by 2-way ANOVAS (stimulus x electrode) conducted separately for the random condition (middle section) and the predictable condition (lower section).}
 \setlength{\tabcolsep}{0pt}
\begin{tabular}{l l l l l l l l l}


\hhline{>{\huxb{0, 0, 0}{0.4}}->{\huxb{0, 0, 0}{0.4}}->{\huxb{0, 0, 0}{0.4}}->{\huxb{0, 0, 0}{0.4}}->{\huxb{0, 0, 0}{0.4}}->{\huxb{0, 0, 0}{0.4}}->{\huxb{0, 0, 0}{0.4}}->{\huxb{0, 0, 0}{0.4}}->{\huxb{0, 0, 0}{0.4}}-}
\arrayrulecolor{black}

\multicolumn{1}{!{\huxvb{0, 0, 0}{0}}l!{\huxvb{0, 0, 0}{0}}}{\huxtpad{6pt + 1em}\raggedright \hspace{0pt} \rotatebox{90}{\textbf{}} \hspace{6pt}\huxbpad{6pt}} &
\multicolumn{1}{l!{\huxvb{0, 0, 0}{0}}}{\huxtpad{6pt + 1em}\raggedright \hspace{6pt} \rotatebox{90}{\textbf{}} \hspace{6pt}\huxbpad{6pt}} &
\multicolumn{1}{l!{\huxvb{0, 0, 0}{0}}}{\huxtpad{6pt + 1em}\raggedright \hspace{6pt} \textbf{Effect} \hspace{6pt}\huxbpad{6pt}} &
\multicolumn{1}{r!{\huxvb{0, 0, 0}{0}}}{\huxtpad{6pt + 1em}\raggedleft \hspace{6pt} \textbf{DFn} \hspace{6pt}\huxbpad{6pt}} &
\multicolumn{1}{r!{\huxvb{0, 0, 0}{0}}}{\huxtpad{6pt + 1em}\raggedleft \hspace{6pt} \textbf{DFd} \hspace{6pt}\huxbpad{6pt}} &
\multicolumn{1}{r!{\huxvb{0, 0, 0}{0}}}{\huxtpad{6pt + 1em}\raggedleft \hspace{6pt} \textbf{F} \hspace{6pt}\huxbpad{6pt}} &
\multicolumn{1}{r!{\huxvb{0, 0, 0}{0}}}{\huxtpad{6pt + 1em}\raggedleft \hspace{6pt} \textbf{p} \hspace{6pt}\huxbpad{6pt}} &
\multicolumn{1}{l!{\huxvb{0, 0, 0}{0}}}{\huxtpad{6pt + 1em}\raggedright \hspace{6pt} \textbf{p$<$.05} \hspace{6pt}\huxbpad{6pt}} &
\multicolumn{1}{r!{\huxvb{0, 0, 0}{0}}}{\huxtpad{6pt + 1em}\raggedleft \hspace{6pt} \textbf{ges} \hspace{0pt}\huxbpad{6pt}} \tabularnewline[-0.5pt]


\hhline{>{\huxb{0, 0, 0}{0.4}}->{\huxb{0, 0, 0}{0.4}}->{\huxb{0, 0, 0}{0.4}}->{\huxb{0, 0, 0}{0.4}}->{\huxb{0, 0, 0}{0.4}}->{\huxb{0, 0, 0}{0.4}}->{\huxb{0, 0, 0}{0.4}}->{\huxb{0, 0, 0}{0.4}}->{\huxb{0, 0, 0}{0.4}}-}
\arrayrulecolor{black}

\multicolumn{1}{!{\huxvb{0, 0, 0}{0}}l!{\huxvb{0, 0, 0}{0}}}{} &
\multicolumn{1}{l!{\huxvb{0, 0, 0}{0}}}{} &
\multicolumn{1}{l!{\huxvb{0, 0, 0}{0}}}{\huxtpad{6pt + 1em}\raggedright \hspace{6pt} Condition \hspace{6pt}\huxbpad{6pt}} &
\multicolumn{1}{r!{\huxvb{0, 0, 0}{0}}}{\huxtpad{6pt + 1em}\raggedleft \hspace{6pt} 1 \hspace{6pt}\huxbpad{6pt}} &
\multicolumn{1}{r!{\huxvb{0, 0, 0}{0}}}{\huxtpad{6pt + 1em}\raggedleft \hspace{6pt} 19 \hspace{6pt}\huxbpad{6pt}} &
\multicolumn{1}{r!{\huxvb{0, 0, 0}{0}}}{\huxtpad{6pt + 1em}\raggedleft \hspace{6pt} 0.16~ \hspace{6pt}\huxbpad{6pt}} &
\multicolumn{1}{r!{\huxvb{0, 0, 0}{0}}}{\huxtpad{6pt + 1em}\raggedleft \hspace{6pt} .694 \hspace{6pt}\huxbpad{6pt}} &
\multicolumn{1}{l!{\huxvb{0, 0, 0}{0}}}{\huxtpad{6pt + 1em}\raggedright \hspace{6pt}  \hspace{6pt}\huxbpad{6pt}} &
\multicolumn{1}{r!{\huxvb{0, 0, 0}{0}}}{\huxtpad{6pt + 1em}\raggedleft \hspace{6pt} 0.003~~ \hspace{0pt}\huxbpad{6pt}} \tabularnewline[-0.5pt]


\hhline{}
\arrayrulecolor{black}

\multicolumn{1}{!{\huxvb{0, 0, 0}{0}}l!{\huxvb{0, 0, 0}{0}}}{} &
\multicolumn{1}{l!{\huxvb{0, 0, 0}{0}}}{} &
\multicolumn{1}{l!{\huxvb{0, 0, 0}{0}}}{\huxtpad{6pt + 1em}\raggedright \hspace{6pt} StimulusType \hspace{6pt}\huxbpad{6pt}} &
\multicolumn{1}{r!{\huxvb{0, 0, 0}{0}}}{\huxtpad{6pt + 1em}\raggedleft \hspace{6pt} 1 \hspace{6pt}\huxbpad{6pt}} &
\multicolumn{1}{r!{\huxvb{0, 0, 0}{0}}}{\huxtpad{6pt + 1em}\raggedleft \hspace{6pt} 19 \hspace{6pt}\huxbpad{6pt}} &
\multicolumn{1}{r!{\huxvb{0, 0, 0}{0}}}{\huxtpad{6pt + 1em}\raggedleft \hspace{6pt} 0.006 \hspace{6pt}\huxbpad{6pt}} &
\multicolumn{1}{r!{\huxvb{0, 0, 0}{0}}}{\huxtpad{6pt + 1em}\raggedleft \hspace{6pt} .938 \hspace{6pt}\huxbpad{6pt}} &
\multicolumn{1}{l!{\huxvb{0, 0, 0}{0}}}{\huxtpad{6pt + 1em}\raggedright \hspace{6pt}  \hspace{6pt}\huxbpad{6pt}} &
\multicolumn{1}{r!{\huxvb{0, 0, 0}{0}}}{\huxtpad{6pt + 1em}\raggedleft \hspace{6pt} 1.5e-05 \hspace{0pt}\huxbpad{6pt}} \tabularnewline[-0.5pt]


\hhline{}
\arrayrulecolor{black}

\multicolumn{1}{!{\huxvb{0, 0, 0}{0}}l!{\huxvb{0, 0, 0}{0}}}{} &
\multicolumn{1}{l!{\huxvb{0, 0, 0}{0}}}{\multirow[c]{-3}{*}[0ex]{\huxtpad{6pt + 1em}\raggedright \hspace{6pt} \rotatebox{90}{Frontal} \hspace{6pt}\huxbpad{6pt}}} &
\multicolumn{1}{l!{\huxvb{0, 0, 0}{0}}}{\huxtpad{6pt + 1em}\raggedright \hspace{6pt} Condition x StimulusType \hspace{6pt}\huxbpad{8pt}} &
\multicolumn{1}{r!{\huxvb{0, 0, 0}{0}}}{\huxtpad{6pt + 1em}\raggedleft \hspace{6pt} 1 \hspace{6pt}\huxbpad{8pt}} &
\multicolumn{1}{r!{\huxvb{0, 0, 0}{0}}}{\huxtpad{6pt + 1em}\raggedleft \hspace{6pt} 19 \hspace{6pt}\huxbpad{8pt}} &
\multicolumn{1}{r!{\huxvb{0, 0, 0}{0}}}{\huxtpad{6pt + 1em}\raggedleft \hspace{6pt} 16.7~~ \hspace{6pt}\huxbpad{8pt}} &
\multicolumn{1}{r!{\huxvb{0, 0, 0}{0}}}{\huxtpad{6pt + 1em}\raggedleft \hspace{6pt} $\backslash$$<$ .001 \hspace{6pt}\huxbpad{8pt}} &
\multicolumn{1}{l!{\huxvb{0, 0, 0}{0}}}{\huxtpad{6pt + 1em}\raggedright \hspace{6pt} * \hspace{6pt}\huxbpad{8pt}} &
\multicolumn{1}{r!{\huxvb{0, 0, 0}{0}}}{\huxtpad{6pt + 1em}\raggedleft \hspace{6pt} 0.013~~ \hspace{0pt}\huxbpad{8pt}} \tabularnewline[-0.5pt]


\hhline{}
\arrayrulecolor{black}

\multicolumn{1}{!{\huxvb{0, 0, 0}{0}}l!{\huxvb{0, 0, 0}{0}}}{} &
\multicolumn{1}{l!{\huxvb{0, 0, 0}{0}}}{} &
\multicolumn{1}{l!{\huxvb{0, 0, 0}{0}}}{\huxtpad{8pt + 1em}\raggedright \hspace{6pt} Condition \hspace{6pt}\huxbpad{6pt}} &
\multicolumn{1}{r!{\huxvb{0, 0, 0}{0}}}{\huxtpad{8pt + 1em}\raggedleft \hspace{6pt} 1 \hspace{6pt}\huxbpad{6pt}} &
\multicolumn{1}{r!{\huxvb{0, 0, 0}{0}}}{\huxtpad{8pt + 1em}\raggedleft \hspace{6pt} 19 \hspace{6pt}\huxbpad{6pt}} &
\multicolumn{1}{r!{\huxvb{0, 0, 0}{0}}}{\huxtpad{8pt + 1em}\raggedleft \hspace{6pt} 1.28~ \hspace{6pt}\huxbpad{6pt}} &
\multicolumn{1}{r!{\huxvb{0, 0, 0}{0}}}{\huxtpad{8pt + 1em}\raggedleft \hspace{6pt} .272 \hspace{6pt}\huxbpad{6pt}} &
\multicolumn{1}{l!{\huxvb{0, 0, 0}{0}}}{\huxtpad{8pt + 1em}\raggedright \hspace{6pt}  \hspace{6pt}\huxbpad{6pt}} &
\multicolumn{1}{r!{\huxvb{0, 0, 0}{0}}}{\huxtpad{8pt + 1em}\raggedleft \hspace{6pt} 0.014~~ \hspace{0pt}\huxbpad{6pt}} \tabularnewline[-0.5pt]


\hhline{}
\arrayrulecolor{black}

\multicolumn{1}{!{\huxvb{0, 0, 0}{0}}l!{\huxvb{0, 0, 0}{0}}}{} &
\multicolumn{1}{l!{\huxvb{0, 0, 0}{0}}}{} &
\multicolumn{1}{l!{\huxvb{0, 0, 0}{0}}}{\huxtpad{6pt + 1em}\raggedright \hspace{6pt} StimulusType \hspace{6pt}\huxbpad{6pt}} &
\multicolumn{1}{r!{\huxvb{0, 0, 0}{0}}}{\huxtpad{6pt + 1em}\raggedleft \hspace{6pt} 1 \hspace{6pt}\huxbpad{6pt}} &
\multicolumn{1}{r!{\huxvb{0, 0, 0}{0}}}{\huxtpad{6pt + 1em}\raggedleft \hspace{6pt} 19 \hspace{6pt}\huxbpad{6pt}} &
\multicolumn{1}{r!{\huxvb{0, 0, 0}{0}}}{\huxtpad{6pt + 1em}\raggedleft \hspace{6pt} 1.21~ \hspace{6pt}\huxbpad{6pt}} &
\multicolumn{1}{r!{\huxvb{0, 0, 0}{0}}}{\huxtpad{6pt + 1em}\raggedleft \hspace{6pt} .285 \hspace{6pt}\huxbpad{6pt}} &
\multicolumn{1}{l!{\huxvb{0, 0, 0}{0}}}{\huxtpad{6pt + 1em}\raggedright \hspace{6pt}  \hspace{6pt}\huxbpad{6pt}} &
\multicolumn{1}{r!{\huxvb{0, 0, 0}{0}}}{\huxtpad{6pt + 1em}\raggedleft \hspace{6pt} 0.004~~ \hspace{0pt}\huxbpad{6pt}} \tabularnewline[-0.5pt]


\hhline{}
\arrayrulecolor{black}

\multicolumn{1}{!{\huxvb{0, 0, 0}{0}}l!{\huxvb{0, 0, 0}{0}}}{\multirow[c]{-6}{*}[0ex]{\huxtpad{6pt + 1em}\raggedright \hspace{0pt} \rotatebox{90}{100 ms} \hspace{6pt}\huxbpad{6pt}}} &
\multicolumn{1}{l!{\huxvb{0, 0, 0}{0}}}{\multirow[c]{-3}{*}[0ex]{\huxtpad{8pt + 1em}\raggedright \hspace{6pt} \rotatebox{90}{Mastoids} \hspace{6pt}\huxbpad{6pt}}} &
\multicolumn{1}{l!{\huxvb{0, 0, 0}{0}}}{\huxtpad{6pt + 1em}\raggedright \hspace{6pt} Condition x StimulusType \hspace{6pt}\huxbpad{15pt}} &
\multicolumn{1}{r!{\huxvb{0, 0, 0}{0}}}{\huxtpad{6pt + 1em}\raggedleft \hspace{6pt} 1 \hspace{6pt}\huxbpad{15pt}} &
\multicolumn{1}{r!{\huxvb{0, 0, 0}{0}}}{\huxtpad{6pt + 1em}\raggedleft \hspace{6pt} 19 \hspace{6pt}\huxbpad{15pt}} &
\multicolumn{1}{r!{\huxvb{0, 0, 0}{0}}}{\huxtpad{6pt + 1em}\raggedleft \hspace{6pt} 2.37~ \hspace{6pt}\huxbpad{15pt}} &
\multicolumn{1}{r!{\huxvb{0, 0, 0}{0}}}{\huxtpad{6pt + 1em}\raggedleft \hspace{6pt} .141 \hspace{6pt}\huxbpad{15pt}} &
\multicolumn{1}{l!{\huxvb{0, 0, 0}{0}}}{\huxtpad{6pt + 1em}\raggedright \hspace{6pt}  \hspace{6pt}\huxbpad{15pt}} &
\multicolumn{1}{r!{\huxvb{0, 0, 0}{0}}}{\huxtpad{6pt + 1em}\raggedleft \hspace{6pt} 0.009~~ \hspace{0pt}\huxbpad{15pt}} \tabularnewline[-0.5pt]


\hhline{}
\arrayrulecolor{black}

\multicolumn{1}{!{\huxvb{0, 0, 0}{0}}l!{\huxvb{0, 0, 0}{0}}}{} &
\multicolumn{1}{l!{\huxvb{0, 0, 0}{0}}}{} &
\multicolumn{1}{l!{\huxvb{0, 0, 0}{0}}}{\huxtpad{15pt + 1em}\raggedright \hspace{6pt} Condition \hspace{6pt}\huxbpad{6pt}} &
\multicolumn{1}{r!{\huxvb{0, 0, 0}{0}}}{\huxtpad{15pt + 1em}\raggedleft \hspace{6pt} 1 \hspace{6pt}\huxbpad{6pt}} &
\multicolumn{1}{r!{\huxvb{0, 0, 0}{0}}}{\huxtpad{15pt + 1em}\raggedleft \hspace{6pt} 22 \hspace{6pt}\huxbpad{6pt}} &
\multicolumn{1}{r!{\huxvb{0, 0, 0}{0}}}{\huxtpad{15pt + 1em}\raggedleft \hspace{6pt} 0.947 \hspace{6pt}\huxbpad{6pt}} &
\multicolumn{1}{r!{\huxvb{0, 0, 0}{0}}}{\huxtpad{15pt + 1em}\raggedleft \hspace{6pt} .341 \hspace{6pt}\huxbpad{6pt}} &
\multicolumn{1}{l!{\huxvb{0, 0, 0}{0}}}{\huxtpad{15pt + 1em}\raggedright \hspace{6pt}  \hspace{6pt}\huxbpad{6pt}} &
\multicolumn{1}{r!{\huxvb{0, 0, 0}{0}}}{\huxtpad{15pt + 1em}\raggedleft \hspace{6pt} 0.006~~ \hspace{0pt}\huxbpad{6pt}} \tabularnewline[-0.5pt]


\hhline{}
\arrayrulecolor{black}

\multicolumn{1}{!{\huxvb{0, 0, 0}{0}}l!{\huxvb{0, 0, 0}{0}}}{} &
\multicolumn{1}{l!{\huxvb{0, 0, 0}{0}}}{} &
\multicolumn{1}{l!{\huxvb{0, 0, 0}{0}}}{\huxtpad{6pt + 1em}\raggedright \hspace{6pt} StimulusType \hspace{6pt}\huxbpad{6pt}} &
\multicolumn{1}{r!{\huxvb{0, 0, 0}{0}}}{\huxtpad{6pt + 1em}\raggedleft \hspace{6pt} 1 \hspace{6pt}\huxbpad{6pt}} &
\multicolumn{1}{r!{\huxvb{0, 0, 0}{0}}}{\huxtpad{6pt + 1em}\raggedleft \hspace{6pt} 22 \hspace{6pt}\huxbpad{6pt}} &
\multicolumn{1}{r!{\huxvb{0, 0, 0}{0}}}{\huxtpad{6pt + 1em}\raggedleft \hspace{6pt} 22.7~~ \hspace{6pt}\huxbpad{6pt}} &
\multicolumn{1}{r!{\huxvb{0, 0, 0}{0}}}{\huxtpad{6pt + 1em}\raggedleft \hspace{6pt} $\backslash$$<$ .001 \hspace{6pt}\huxbpad{6pt}} &
\multicolumn{1}{l!{\huxvb{0, 0, 0}{0}}}{\huxtpad{6pt + 1em}\raggedright \hspace{6pt} * \hspace{6pt}\huxbpad{6pt}} &
\multicolumn{1}{r!{\huxvb{0, 0, 0}{0}}}{\huxtpad{6pt + 1em}\raggedleft \hspace{6pt} 0.038~~ \hspace{0pt}\huxbpad{6pt}} \tabularnewline[-0.5pt]


\hhline{}
\arrayrulecolor{black}

\multicolumn{1}{!{\huxvb{0, 0, 0}{0}}l!{\huxvb{0, 0, 0}{0}}}{} &
\multicolumn{1}{l!{\huxvb{0, 0, 0}{0}}}{\multirow[c]{-3}{*}[0ex]{\huxtpad{15pt + 1em}\raggedright \hspace{6pt} \rotatebox{90}{Frontal} \hspace{6pt}\huxbpad{6pt}}} &
\multicolumn{1}{l!{\huxvb{0, 0, 0}{0}}}{\huxtpad{6pt + 1em}\raggedright \hspace{6pt} Condition x StimulusType \hspace{6pt}\huxbpad{8pt}} &
\multicolumn{1}{r!{\huxvb{0, 0, 0}{0}}}{\huxtpad{6pt + 1em}\raggedleft \hspace{6pt} 1 \hspace{6pt}\huxbpad{8pt}} &
\multicolumn{1}{r!{\huxvb{0, 0, 0}{0}}}{\huxtpad{6pt + 1em}\raggedleft \hspace{6pt} 22 \hspace{6pt}\huxbpad{8pt}} &
\multicolumn{1}{r!{\huxvb{0, 0, 0}{0}}}{\huxtpad{6pt + 1em}\raggedleft \hspace{6pt} 0.028 \hspace{6pt}\huxbpad{8pt}} &
\multicolumn{1}{r!{\huxvb{0, 0, 0}{0}}}{\huxtpad{6pt + 1em}\raggedleft \hspace{6pt} .868 \hspace{6pt}\huxbpad{8pt}} &
\multicolumn{1}{l!{\huxvb{0, 0, 0}{0}}}{\huxtpad{6pt + 1em}\raggedright \hspace{6pt}  \hspace{6pt}\huxbpad{8pt}} &
\multicolumn{1}{r!{\huxvb{0, 0, 0}{0}}}{\huxtpad{6pt + 1em}\raggedleft \hspace{6pt} 2.2e-05 \hspace{0pt}\huxbpad{8pt}} \tabularnewline[-0.5pt]


\hhline{}
\arrayrulecolor{black}

\multicolumn{1}{!{\huxvb{0, 0, 0}{0}}l!{\huxvb{0, 0, 0}{0}}}{} &
\multicolumn{1}{l!{\huxvb{0, 0, 0}{0}}}{} &
\multicolumn{1}{l!{\huxvb{0, 0, 0}{0}}}{\huxtpad{8pt + 1em}\raggedright \hspace{6pt} Condition \hspace{6pt}\huxbpad{6pt}} &
\multicolumn{1}{r!{\huxvb{0, 0, 0}{0}}}{\huxtpad{8pt + 1em}\raggedleft \hspace{6pt} 1 \hspace{6pt}\huxbpad{6pt}} &
\multicolumn{1}{r!{\huxvb{0, 0, 0}{0}}}{\huxtpad{8pt + 1em}\raggedleft \hspace{6pt} 22 \hspace{6pt}\huxbpad{6pt}} &
\multicolumn{1}{r!{\huxvb{0, 0, 0}{0}}}{\huxtpad{8pt + 1em}\raggedleft \hspace{6pt} 0.206 \hspace{6pt}\huxbpad{6pt}} &
\multicolumn{1}{r!{\huxvb{0, 0, 0}{0}}}{\huxtpad{8pt + 1em}\raggedleft \hspace{6pt} .655 \hspace{6pt}\huxbpad{6pt}} &
\multicolumn{1}{l!{\huxvb{0, 0, 0}{0}}}{\huxtpad{8pt + 1em}\raggedright \hspace{6pt}  \hspace{6pt}\huxbpad{6pt}} &
\multicolumn{1}{r!{\huxvb{0, 0, 0}{0}}}{\huxtpad{8pt + 1em}\raggedleft \hspace{6pt} 0.001~~ \hspace{0pt}\huxbpad{6pt}} \tabularnewline[-0.5pt]


\hhline{}
\arrayrulecolor{black}

\multicolumn{1}{!{\huxvb{0, 0, 0}{0}}l!{\huxvb{0, 0, 0}{0}}}{} &
\multicolumn{1}{l!{\huxvb{0, 0, 0}{0}}}{} &
\multicolumn{1}{l!{\huxvb{0, 0, 0}{0}}}{\huxtpad{6pt + 1em}\raggedright \hspace{6pt} StimulusType \hspace{6pt}\huxbpad{6pt}} &
\multicolumn{1}{r!{\huxvb{0, 0, 0}{0}}}{\huxtpad{6pt + 1em}\raggedleft \hspace{6pt} 1 \hspace{6pt}\huxbpad{6pt}} &
\multicolumn{1}{r!{\huxvb{0, 0, 0}{0}}}{\huxtpad{6pt + 1em}\raggedleft \hspace{6pt} 22 \hspace{6pt}\huxbpad{6pt}} &
\multicolumn{1}{r!{\huxvb{0, 0, 0}{0}}}{\huxtpad{6pt + 1em}\raggedleft \hspace{6pt} 6.56~ \hspace{6pt}\huxbpad{6pt}} &
\multicolumn{1}{r!{\huxvb{0, 0, 0}{0}}}{\huxtpad{6pt + 1em}\raggedleft \hspace{6pt} .018 \hspace{6pt}\huxbpad{6pt}} &
\multicolumn{1}{l!{\huxvb{0, 0, 0}{0}}}{\huxtpad{6pt + 1em}\raggedright \hspace{6pt} * \hspace{6pt}\huxbpad{6pt}} &
\multicolumn{1}{r!{\huxvb{0, 0, 0}{0}}}{\huxtpad{6pt + 1em}\raggedleft \hspace{6pt} 0.018~~ \hspace{0pt}\huxbpad{6pt}} \tabularnewline[-0.5pt]


\hhline{}
\arrayrulecolor{black}

\multicolumn{1}{!{\huxvb{0, 0, 0}{0}}l!{\huxvb{0, 0, 0}{0}}}{\multirow[c]{-6}{*}[0ex]{\huxtpad{15pt + 1em}\raggedright \hspace{0pt} \rotatebox{90}{150 ms} \hspace{6pt}\huxbpad{6pt}}} &
\multicolumn{1}{l!{\huxvb{0, 0, 0}{0}}}{\multirow[c]{-3}{*}[0ex]{\huxtpad{8pt + 1em}\raggedright \hspace{6pt} \rotatebox{90}{Mastoids} \hspace{6pt}\huxbpad{6pt}}} &
\multicolumn{1}{l!{\huxvb{0, 0, 0}{0}}}{\huxtpad{6pt + 1em}\raggedright \hspace{6pt} Condition x StimulusType \hspace{6pt}\huxbpad{6pt}} &
\multicolumn{1}{r!{\huxvb{0, 0, 0}{0}}}{\huxtpad{6pt + 1em}\raggedleft \hspace{6pt} 1 \hspace{6pt}\huxbpad{6pt}} &
\multicolumn{1}{r!{\huxvb{0, 0, 0}{0}}}{\huxtpad{6pt + 1em}\raggedleft \hspace{6pt} 22 \hspace{6pt}\huxbpad{6pt}} &
\multicolumn{1}{r!{\huxvb{0, 0, 0}{0}}}{\huxtpad{6pt + 1em}\raggedleft \hspace{6pt} 0.122 \hspace{6pt}\huxbpad{6pt}} &
\multicolumn{1}{r!{\huxvb{0, 0, 0}{0}}}{\huxtpad{6pt + 1em}\raggedleft \hspace{6pt} .730 \hspace{6pt}\huxbpad{6pt}} &
\multicolumn{1}{l!{\huxvb{0, 0, 0}{0}}}{\huxtpad{6pt + 1em}\raggedright \hspace{6pt}  \hspace{6pt}\huxbpad{6pt}} &
\multicolumn{1}{r!{\huxvb{0, 0, 0}{0}}}{\huxtpad{6pt + 1em}\raggedleft \hspace{6pt} 0.00028 \hspace{0pt}\huxbpad{6pt}} \tabularnewline[-0.5pt]


\hhline{>{\huxb{0, 0, 0}{0.4}}->{\huxb{0, 0, 0}{0.4}}->{\huxb{0, 0, 0}{0.4}}->{\huxb{0, 0, 0}{0.4}}->{\huxb{0, 0, 0}{0.4}}->{\huxb{0, 0, 0}{0.4}}->{\huxb{0, 0, 0}{0.4}}->{\huxb{0, 0, 0}{0.4}}->{\huxb{0, 0, 0}{0.4}}-}
\arrayrulecolor{black}
\end{tabular}
\end{threeparttable}\par\end{centerbox}

\end{table}



Post-hoc tests between ERPs to A and B tones were carried out using
paired Student's t-Tests. P-values were corrected for multiple
comparisons using the Benjamini--Hochberg step-up procedure. For the 100
ms SOA, results indicate a significant effect only for predictable tones
at fronto-central electrodes (\(t(19) = -2.77\), \(p_{adj} = p = .025\),
\(d = -0.62\)). For the 150 ms SOA, B tones elicted significantly more
negative ERPs than B tones at fronto-central electrode locations in both
the predictable (\(t(22) = 5.20\), \(p_{adj} = p lt .001\),
\(d = 1.08\)) and random (\(t(22) = 3.28\), \(p_{adj} = p = .009\),
\(d = 0.68\)) conditions. Significant polarity reversal effects at
mastoid sites was only present dor predcitable tones (\(t(22) = -3.95\),
\(p_{adj} = p = .003\), \(d = -0.82\)) but not for randomly presented
tones (\(t(22) = -1.59\), \(p_{adj} = p = .169\), \(d = -0.33\)).

\begin{figure}
\centering
\includegraphics{figures/fig_posthoc.pdf}
\caption{Averaged voltages in the MMN latency window for pooled
fronto-central and mastoid electrodes. Colored areas show sample
probability density function for A tones (green) and B tones (red).
White diamonds indicate estimated population mean, vertical bars
represent 95\%-conficence interval. Only Benjamini-Hochberg-corrected
p-values \(< 0.05\) are shown.}
\end{figure}

Figure X shows EEG waveform averages for five-tone sequences (A-A-A-A-B)
presented in a \emph{predictable} (top panel) and \emph{random} contexts
(lower panel).

\begin{figure}
\centering
\includegraphics{figures/fig_sequences.pdf}
\caption{EEG waveforms for five-tone sequences presented in an
predictable context (dotted line) and pseudo-random condition (dashed
line) for 100 ms presentation rate (top panel) and 150 ms presentation
rate (lower pabel). Vertical lines indicate tone onset.}
\end{figure}

\begin{figure}
\centering
\includegraphics{figures/fig_subsample_rel.pdf}
\caption{EEG waveforms for five-tone sequences presented in an
predictable context (dotted line) and pseudo-random condition (dashed
line) for 100 ms presentation rate (top panel) and 150 ms presentation
rate (lower pabel). Vertical lines indicate tone onset.}
\end{figure}

\newpage

\begin{verbatim}
```{=ms}
# References
\end{verbatim}

\end{document}
